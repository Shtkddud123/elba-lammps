\documentclass[10pt]{article}
%\documentclass[titlepage]{article}

% Packages to load (all standard on a modern LaTeX system on Linux)

\usepackage{setspace} 
\usepackage{wrapfig} 
\usepackage{graphicx}
\usepackage{array}
\usepackage{subfigure}
%\usepackage[super]{cite} 

\usepackage[super,numbers,sort&compress]{natbib} 
%\usepackage[super,sort&compress]{plain} 
% Bibliography style (requires the style file biophysj.bst in the 
% document directory)
\bibliographystyle{jpc}
% Numbering style in the list of references: a number followed by a period
\renewcommand{\bibnumfmt}[1]{#1}
\renewcommand{\thefootnote}{\fnsymbol{footnote}}

\setlength{\oddsidemargin}{-0.35in} 
\setlength{\evensidemargin}{-0.05in} 
\setlength{\textwidth}{7.1in}
\setlength{\headheight}{-1.in} 
\setlength{\topmargin}{.5in} % decrease to optimise space  
\setlength{\textheight}{10.2in}
%\setlength{\headheight}{-0.2in} 
%\setlength{\topmargin}{-0.5in}  
%\setlength{\textheight}{10in}


\title{ELBA forcefield}

% Revision date - uncomment to exclude date in the final version
\date{}

\begin{document}
\large{
\bf{ELBA-1.0}}
\medskip
\begin{table*}[!ht]
\begin{tabular}{|c|c|c|c|}
Symbol & BRAHMS units & LAMMPS units & Other units \\
$\sigma_{CC}$, $\sigma_{PP}$ & 0.52\,nm & 5.2\,\AA&\\
$\sigma_{GG}$, $\sigma_{EE}$ & 0.46\,nm & 4.6\,\AA&\\
$\sigma_{AA}$, $\sigma_{TT}$ & 0.45\,nm & 4.5\,\AA&\\
$\sigma_{WW}$ & 0.30\,nm & 3.0\,\AA&\\
$\epsilon_{CC}$, $\epsilon_{PP}$ & 6.0\,kJ/mol & 1.434\,kcal/mol&\\
$\epsilon_{GG}$, $\epsilon_{EE}$ & 4.0\,kJ/mol & 0.956\,kcal/mol&\\
$\epsilon_{AA}$, $\epsilon_{TT}$  & 3.5\,kJ/mol & 0.837\,kcal/mol&\\
$\epsilon_{WW}$  & 1\,kJ/mol & 0.239\,kcal/mol&\\
$\epsilon^{tot}_{WW}$  & $1.95\,\epsilon_{WW}$ & 0.466\,kcal/mol&\\
$\epsilon_{WP}$ &$1.8\sqrt{\epsilon_{WW}\,\epsilon_{PP}} $ & 1.054\,kcal/mol&\\
$\epsilon_{WG}$ & $1.2\sqrt{\epsilon_{WW}\,\epsilon_{GG}} $ & 0.574\,kcal/mol&\\
$\epsilon_{WE}$  & $1.6\sqrt{\epsilon_{WW}\,\epsilon_{EE}} $ & 0.765\,kcal/mol&\\
$\epsilon_{AW}$  & $1.5\sqrt{\epsilon_{AA}\,\epsilon_{WW}} $ & 0.671\,kcal/mol&\\
$\epsilon_{AP}$  & $2.5\sqrt{\epsilon_{AA}\,\epsilon_{PP}} $ & 2.738\,kcal/mol&\\
$\epsilon_{AG}$  & $1.5\sqrt{\epsilon_{AA}\,\epsilon_{GG}} $ & 1.341\,kcal/mol&\\
$\epsilon_{AE}$  & $2.5\sqrt{\epsilon_{AA}\,\epsilon_{EE}} $ & 2.236\,kcal/mol&\\
 $Q_C$, $Q_A$   &$+0.7$\,e &$+0.7$\,e & $1.12\times10^{-19}$\,C\\
 $Q_P$  &$-0.7$\,e &$-0.7$\,e & $-1.12\times10^{-19}$\,C\\
 $\mu_G$ & $1.6$\,D & 0.333\,e\,\AA&\\
 $\mu_E$ & $2$\,D & 0.416\,e\,\AA&\\
 $\mu_W$ &$2.3$\,D & 0.479\,e\,\AA&\\
 $k$& 1260\,kJ/(mol\,nm$^{2}$)& 3.011\,kcal/(mol\,\AA$^2$)&\\
 $w$& 30\,kJ/mol& 2*3.585\,kcal/mol&\\
 $c$& 10\,kJ/mol& 2*1.195\,kcal/mol&\\
 $\alpha_{0_{CPG}}$, $\alpha_{0_{APG}}$ & $115^{\circ}$& $115^{\circ}$& $2.01$\,rad\\
 $\alpha_{0_{PGE}}$ & $160^{\circ}$&$160^{\circ}$&$2.79$\,rad\\
 $\alpha_{0_{GET}},\alpha_{0_{ETT}},\alpha_{0_{TTT}}^{saturated}$ & $180^{\circ}$& $180^{\circ}$&$3.14$\,rad\\
 $\alpha_{0_{TTT}}^{cis-unsaturated}$ & $120^{\circ}$ & $120^{\circ}$ &$2.09$\,rad\\
 $m_C, m_P$ &  90\,amu&90\,g/mol& $14.9\times10^{-23}$\,g\\ 
 $m_G, m_E$ &  62\,amu& 62\,g/mol& $10.3\times10^{-23}$\,g\\
 $m_T$, $m_A$ &  42\,amu &  42\,g/mol & $7.0\times10^{-23}$\,g\\
 $m_W$ &  40\,amu &40\,g/mol & $6.6\times10^{-23}$\,g\\
 $I_G, _E$ &  10\,amu\,nm$^{2}$& & 1000\,amu\,\AA$^2$\\ 
 $I_W$ &  1\,amu\,nm$^{2}$& & 100\,amu\,\AA$^2$\\
\end{tabular}
\begin{flushleft}Subscripts $C$, $A$, $P$, $G$, $E$, $T$ and $W$ stand for the site types {\it choline}, {\it amine}, {\it phosphate}, {\it glycerol}, {\it ester}, {\it tail} and {\it water}, respectively. Lennard-Jones cross-terms are calculated by the standard Lorentz-Berthelot rule %~\cite{allen} 
except for increased $\epsilon$ terms representing hydrogen bonding; in particular, $\epsilon_{WP},\epsilon_{WG}, \epsilon_{WE}$, $\epsilon_{AW}$, $\epsilon_{AP}$, $\epsilon_{AG}$, $\epsilon_{AE}$ are set as reported in the table. Charges and dipoles are identified by $Q$ and $\mu$; cross terms are obtained via standard electrostatic formulae. %~\cite{price84a}.
Nonbonded interactions between bonded pairs are zero (in technical jargon: 1-2 LJ and Coul interactions are excluded).
The rigidity of the Hooke harmonic potential %(equation~\ref{eq:hooke}) 
is identified by $k$;  reference lengths are set to $0.9\,\sigma_{ij}$. The rigidity of the angle potential %(equation~\ref{eq:ang}) 
is identified by $w$. The rigidity of the orientation-restraining potential %(equation~\ref{eq:orientRestrain}) 
is $c$. Masses and principal moments of inertia are identified by $m$ and $I$, respectively. 
\end{flushleft}
\label{tab:lipidPars}
 \end{table*}


%\doublespacing

%\emph{Keywords:}  membrane biophysics;  molecular modelling;  lipid bilayers; membrane proteins; lipid biosynthesis;  anaesthesia; mechanosensitivity;  drug design; nanotechnology; biocompatible materials; drug delivery technology;

% 200 words max Abstract

%\end{document}

\end{document}

